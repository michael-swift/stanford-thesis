\chapter{Concluding Remarks}
In this thesis work, I used a number of new technologies to better understand the dynamics of cellular behavoirs in the immune system. In the first work, I helped create one of the most comprehensive references to date of the molecular characteristics of human cells. In the second work, I used these new tools to comprehensively investigate B cell the molecular characteristics of their activation and differentiation. This work combined an old approach, lineage tracing, with the new single-cell transcriptomic approaches to systematically understand how clones behave in response to external stimuli. Paradoxically, I view this work as a simple proof-of-concept. The technologies I used do not yet allow the scale necessary to explore multiple axes of B cell differentiation, such as how B cells from other tissue besides the peripheral blood would respond to the same stimulus, or what the differentiation process looks like in response to a large panel of stimuli. Nonetheless, we quantitatively characterized the extent to which B cell clones propagate information about their initial states while they remodel their phentoypes in response to stimuli. In my opinion, the most important result is that even with strong over-clustering of the neighborhood graphs of transcriptomes, we see clones occupy more similar regions of the space than expected by random chance. This suggests that cell types or states are more granular in a meaningful way, than researchers had previously thought. The approaches developed there should only become more useful as the scale of these measurements increase. In the third work, we I developed a dataset which shows the B cell repertoire at unprecedented resolution. However, I like to jokingly call it an 8-bit representation of the antibody repertoire. Once again, due to scaling constraints, this is a grainy picture of the immune system. From this picture, we were able to infer key statistical properties of the human immune system, such as how much sharing of clones occurs between tissues. In addition, broadly speaking, we were able to show how B cell fate decisions are made \textit{in vivo}, and identify ever more granular sub-types of B cells.
Single cell approaches have long been a dream of biologists, as cells are the fundamental unit of biology. Indeed, the molecular biology revolution of the 1950s and 1960s was kicked-off by clever experiments which counted the progeny of single bacterial cell or phages with mutations. Now with powerful single-cell tools, I have helped create detailed pictures of the molecular contents of the cells in the body. These references will prove useful for many tasks, including diagnostic development and inferences into possible targets for therapeutics. In some ways, it appears biology is going the way of physics, where data is generated by massive consortia in multi-year, multi-million dollar projects. The trend towards making biology a data-driven science is good. However, I believe the most important advances will come from people with incisive lines of questioning, working in small groups towards the answers. New single-cell tools will never substitute for framing the right questions and designing the right experiments. Incisive follow-on investigations, will be aided by these atlasing and profiling efforts, but focused work on fundamental questions of general interest is more important than ever. The next suite of biological tools are the ones are likely to create the next biological revolution if they can combine the unbiased data-generation exemplified here with facile re-programming of cells.      
