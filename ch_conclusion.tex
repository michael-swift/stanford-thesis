\chapter{Concluding Remarks}
In this thesis, I have employed a variety of emerging technologies to gain a deeper understanding of the dynamics of cellular behaviors within the immune system.

In the first work, I participated in the development of one of the most comprehensive references to date on the molecular characteristics of human cells. Subsequently, in the second work, I utilized these novel tools to thoroughly investigate the molecular characteristics of B cell activation and differentiation. This effort combined a traditional approach, lineage tracing, with contemporary single-cell transcriptomic methods to systematically comprehend how clones react to external stimuli. Although I consider this work a proof-of-concept, it is worth noting that the technologies used have not yet scaled sufficiently to explore the full spectrum of B cell differentiation. Uncertainties still exist regarding how B cells from other tissues respond to the same stimulus or what their responses to a broad, systematically designed panel of stimuli might be.

Nonetheless, we have quantitatively characterized the degree to which B cell clones retain information about their initial transcriptional states as they adapt their phenotypes in response to stimuli. This effect was so profound that we noted related cells maintained their transcriptomic associations even after extended periods of differentiation in culture. Additionally, we contributed ideas on how to conduct population tracking in cell reprogramming experiments. The approaches developed in this work are expected to become even more beneficial as the scale of these measurements increases.

In the third work, in collaboration with Ivana Cvijovic, I assembled a dataset providing an unprecedentedly detailed view of the B cell repertoire. However, due to scaling limitations, we consider it an 8-bit representation of the antibody repertoire. This means our picture of the immune system, despite its resolution, remains somewhat grainy. From this dataset, we managed to infer key statistical properties of the human immune system, such as the extent of clone sharing between tissues. Furthermore, we were able to obtain a statistical perspective on the evolutionary process of B cell fate decisions \textit{in vivo}.

Single-cell approaches have been a longstanding aspiration for biologists, given that cells are the fundamental unit of biology. Now, with the advent of powerful single-cell tools, I have contributed to the creation of detailed profiles of the molecular contents of the body's cells. These references are expected to be valuable for a range of applications, including diagnostic development and the identification of potential therapeutic targets. In some respects, it seems biology is following the path of physics, where massive consortia produce organism-wide data through multi-year, multi-million dollar projects. Fundamentally, we can consider these tools as a means of 'reading' biology. In my future scientific endeavors, I aim to develop a suite of tools that will enable us to 'write' or 'program' biology on a similar scale\cite{hoose2023dna}.