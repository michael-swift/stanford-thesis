\chapter{Introduction}

\section{Background}

\section{The promise of single-cell biology}

Biology is a watery and messy soup. The cell is mostly a bag of water, with a generous portion of biological macro-molecules like proteins and nucleic acids. These molecules continuously move about in random, riotous motion\cite{berg1993random}. This randomness is statistically reproducible on the cellular and organismal level, which is what allows life to reproduce itself. Ensembles of cells work in unison create functional and successful organisms, such as the one reading this thesis. A key question in biology how these ensembles work. We have some reasonable answers to this question. In the case of human biology, there are around 20,000 genes in the genome. Each cell translates a subset of these genes into proteins and the identity of these proteins is what makes a neuron have a different form and function compared to a blood cell. During development, the identity of these cell types is determined early and is often maintained throughout the life of the organism. While this model is broadly explanatory, long-standing goals of biology are to understand which molecules are expressed in cells, when does expression happen, and how is this maintained. To bring back the bag metaphor: how does this messy bag of water and bio-molecules create an organism with 1000s of distinct bags working autonomous unison? While this question is fascinating \textit{per se}, the benefits of developing a deep understanding of development are immediately practical. A real understanding of these molecular processes would allow us to program biology towards the goal of eliminating human diseases. At this time, a major target for reprogramming efforts is the human immune system, which is in a constant state of development and diversification. Recently, single-cell sequencing technologies have been developed which allow us to take a step towards defining the molecular processes at work in cells by comprehensively measuring which molecules are there. In this thesis, I used these technologies to build a quantitative understanding of the diversity of cell types and proteins in the human immune system. I take steps towards this goal in three related projects. The first project aimed to create a comprehensive parts-list for human biology. Specifically, the cellular bag could be could be abstracted as containing just RNA\cite{quake2021cell}: what's in the bags?  The second project measured the maintenance of B cell identities and bias during cellular reprogramming: if the contents of the bag must change, how is the identity of the bag maintained? The third project aimed to understand the statistics which describe and govern the dynamics of the human B cell repertoire: what governs the decisions of the bags to change their composition and location in the body?     

\section{A parts-list for human biology} 
In the first part of my thesis, I leveraged recent advances in our ability to systematically measure the identities of macro-molecules single cells. This parts-list for biology creates a systematic understanding of the available functions of cells. By analogy, if an organism were a kitchen, we are now able to classify all the appliances therein by enumerating the parts in each. We now know what parts go into the toasters, refrigerators, and blenders of biology. In particular, I used recent advances in single-cell RNA sequencing techniques quantify gene expression and cellular identities at the individual of cells\cite{klein_droplet_2015, macosko2015highly}, which are an ideal level of abstraction for understanding tissue and organismal biology. These technologies have been employed to investigate various normal developmental systems, such as the Zebrafish and Drosophila\cite{wagner_single-cell_2018,ingle2015drosophila}. Additionally, they have been used to understand abnormal organismal biology such as the dynamics of tumor evolution and metastasis\cite{pierson2017single}. As the throughput and amount of single-cells that could be feasibly analyzed increased\cite{linnarsson2016single}, it became possible to imagine a multi-organ reference of the cell types in the human body\cite{regev2017human}. Through the Tabula Sapiens consortium, I helped create one of broadest references\cite{tabula_sapiens_consortium_tabula_2022}, complemented by other more focused efforts\cite{dominguez_conde_cross-tissue_2022}.
\section{Lineage Tracing in during cell reprogramming}
In the second part of my thesis, I used B cells as a model to quantitatively understand the maintenance of cell-intrinsic identities as cells differentiate. Cells respond to extrinsic cues directing them to change their gene expression programs, which may happen during development or an immune response. Lineage tracing is a powerful technique used to track the developmental history of cells, enabling researchers to understand the complex and dynamic processes governing cell fate decisions. This approach is particularly valuable when tracing the lineage of cells can provide insights into the gene regulatory networks and signaling pathways that shape the trajectories of cellular development. Researchers have employed various experimental and computational approaches to perform lineage tracing, such as quantitative imaging, molecular barcoding, and mathematical modeling\cite{alon2006introduction, lauffenburger2000quantitative, munsky2012using}. In the immune system, lineage tracing has been instrumental in deciphering the developmental pathways that give rise to diverse immune cell types, including T cells, B cells, and natural killer (NK) cells. The work of Phil Hodgkin and associated labs has significantly contributed to our understanding of immune cell development and function through the application of lineage tracing techniques\cite{hodgkin2012cell, marchingo2014t}. With the advent of immune repertoire sequencing and single-cell RNA sequencing, researchers can now perform high-resolution clonal lineage tracing, linking cellular differentiation processes to gene expression profiles of lineages of cells\cite{stubbington2017t, horns2020memory}. Immune repertoire sequencing also allows the characterization of the diverse population of B and T cell receptors (BCRs and TCRs, respectively) within an organism\cite{robins2013immunosequencing, georgiou_promise_2014}. In this work, I quantify the strong intrinsic fate biases of different types of B cells to the same stimulus, describe the extent to which transcriptional memory is passed through generations of B cells, and characterize the magnitude of this effect in Bone-Marrow Derived Plasma Cells \textit{in vivo}.
\section{Lineage Tracing the B cell repertoire across tissues}
In the third part of my thesis, I partnered with Ivana Cvijovic on a project to map the statistics of the the human B cell repertoire. These statistics proved useful for building models of how B cell differentiation proceeds in humans. This work builds on the work in Chapter 1, by generating quantitative single-cell genomics dataset for B cell repertoire analysis. We procured immune-rich organs (spleen, lymph nodes, blood, and bone marrow) from otherwise immunologically healthy individuals and processed these cells via single-cell RNA sequencing. Using the lineage information contained in the antibody sequence, we were able to reconstruct lineage histories, and thus the human B cell differentiation process.
\section{Thesis Organization}
The chapters of this thesis focus on research I did on similar but distinct topics, as outlined above. I have arranged each chapter such that it may stand as its own document, much of the work in each chapter is published (chapters 1 and 2) or will be published (chapter 3).  
