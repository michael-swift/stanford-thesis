\prefacesection{Abstract}
This PhD thesis delves into the intricate mechanisms governing cell differentiation, lineage determination, and migration dynamics in the human immune system, particularly in B and T cells. The research is primarily based on three major contributions that employ single-cell transcriptome sequencing, lineage tracing, and immune repertoire sequencing techniques.

The first contribution details work I did to create Tabula Sapiens: a comprehensive human reference atlas that encompasses nearly 500,000 cells across 24 different tissues and organs. By analyzing the molecular characteristics of over 400 cell types and their distribution, the atlas enables the identification of tissue-specific gene expression and clonal distribution of T cells. The study also investigates B cell mutation rates and class-switch recombination in various tissues.

The second contribution investigates the role of cell-intrinsic factors in determining cell fates and differentiation outcomes during in vitro B cell activation. By analyzing lineage relationships and single-cell RNA sequencing measurements, the study uncovers the transcriptional programs underlying cell-intrinsic clonal fate biases and the interaction between extrinsic signals, intrinsic state, and clonal population structure. 

The third contribution explores the dynamics of B cell migration and clonal expansion through immune repertoire sequencing and single-cell RNA sequencing in multiple tissues from six donors. The findings reveal a correlation between lineage size and migration probability, with larger lineages exhibiting a greater likelihood of migration. Moreover, the study sheds light on the sharing of clones and lineages between tissues and the impact of local micro-environmental cues on gene expression in the bone marrow.

By integrating these multi-modal single-cell analyses, this thesis significantly advances our understanding of the complex processes governing cell differentiation, lineage determination, and migration in the human immune system. 