\prefacesection{Abstract}
This PhD thesis delves into the intricate statistical mechanisms governing cell differentiation, lineage determination, and migration dynamics in the human immune system, particularly in B and T cells. The research makes three major contributions, each employing techniques such as single-cell transcriptome sequencing, lineage tracing, and immune repertoire sequencing.

The first contribution details the work I undertook to create Tabula Sapiens: a comprehensive human reference atlas that encompasses nearly 500,000 cells across 24 different tissues and organs. By analyzing the molecular characteristics of over 400 cell types and their distribution, the atlas enables the identification of tissue-specific gene expression and clonal distribution of T cells. The study also investigates B cell mutation rates and class-switch recombination in various tissues. We characterized the human body at a molecular resolution and publicly shared the dataset, which allows biologists around the world to better understand human biology. 

The second contribution investigates the role of cell-intrinsic factors in determining cell fates and differentiation outcomes during in vitro B cell activation. By analyzing lineage relationships and single-cell RNA sequencing measurements, the study uncovers the transcriptional programs underlying cell-intrinsic clonal fate biases and the interaction between extrinsic signals, intrinsic state, and clonal population structure. We found that although the population of B cells adopted diverse states during differentiation, clones had a restricted set of fates available to them; there were two times more single-fate clones than expected given population-level cell-type diversity. This implicated a molecular memory of initial cell states that was propagated through differentiation. Furthermore, we identified the exact genes which had strongest coherence within clones. These genes significantly overlapped known B cell fate determination programs, suggesting the gene expression states which determine cell identity are most robustly controlled on a clonal level. Persistent clonal identities were also observed in human plasma cells from bone marrow, indicating that these transcriptional programs maintain long-term cell identities \textit{in vivo}.

The third contribution uses the same techniques to explore the dynamics of B cell migration and clonal expansion in multiple tissues from six donors. The findings provide the most detailed picture we have of the B cell system in humans. We find a tight scaling between lineage size and migration probability, with larger lineages exhibiting a greater likelihood of migration. We measure the extent to which memory B cells and ASCs re-enter lymph nodes, which was relatively open question in human immunology. We found memory B cell sharing occurs at a very limited rate compared to ASC sharing. Using phylogenetic approaches, we discovered B cells from the same lineage often co-localize within the same tissue. However, when lineages reach a threshold size in the peripheral blood and secondary lymphatic organs (SLOs), they are more likely to be found in other tissue. This indicates that B cell migration from blood and SLOs is a probabilistic process, with cross-tissue sharing occurring when lineages attain sufficient size. In contrast, clones detected in the Bone Marrow were far more private, consistent with its role as a niche for Long-lived Plasma Cells. Finally, we found the probabilities of differentiation into any cell state across hypermutation levels in individual lineages was remarkably uniform, indicating negligible changes in B cell differentiation potential as they evolve or age. Collectively, our findings provide insights into the statistics of B cell evolution and immune memory formation across various tissues.

By integrating these multi-modal single-cell analyses, this thesis significantly enhances our understanding. It provides new insights into the complex processes governing cell differentiation, lineage determination, and migration in the human immune system. 